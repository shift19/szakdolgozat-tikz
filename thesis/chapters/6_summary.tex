\Chapter{Összefoglalás}

A szakdolgozatom célja egy olyan online grafikus szerkesztő létrehozása volt, amellyel \LaTeX-be visszailleszthető kód hozható létre. 

A dolgozat elkészítése közben alaposabban megismerhettem a Bootstrap 5 \cite{bootstrap} eszköztárral, a KaTeX \cite{katex}, és a \textit{p5.js} \cite{p5js} függvénykönyvtárak működését, előnyeit, hátrányait, és hiányosságait. Ha újra abban a helyzetben lennék, hogy válogatni kellene a könyvtárak közül, biztos vagyok benne, hogy ugyanúgy ezeket választanám. A különböző könyvtárak dokumentációja az online felületükön is elérhetők, interaktív példákkal illusztrálva a működésüket.

Véleményem szerint sikerült egy olyan webes alkalmazást létrehoznom, amely beleillik a mai modern weboldalakba. A célom az volt, hogy egy egyszerű, könnyen használható, azonban mégis funkciókban gazdag alkalmazás készüljön el. Az alkalmazás jelen formájában kiszolgálja egy átlag felhasználó igényeit: tud másolni, beilleszteni és törölni, lehet pontokat mozgatni, valamint meglévő ábrák tulajdonságait módosítani. A már kész ábrák visszatölthetők későbbi szerkesztésre.

A szoftverfejlesztésben még mindig rengeteg lehetőség és esély rejlik. Az alkalmazásom esetében az egyik, amit kiemelnék ezek közül a \LaTeX\ forráskód betöltése, ugyanis jelenleg minden elkészült ábrához tartozik egy kódolt karaktersorozat, amely tartalmazza az információkat, melyből az adott alakzat visszatölthető. Jelenleg a pontok mozgatása kissé körülményes tud lenni nagyobb és részletesebb ábrák esetében, illetve egy teljes alakzat mozgatásához az összes pontját ki kell jelölni. A pontok mozgatásánál jelenleg minden lerakott ábra összes pontja megjelenik, a jövőben célszerű lenne először egy kijelölésre leszűkítve megjeleníteni ezeket. Az alakzatok mozgatása esetében pedig célravezető, ha egy pont kijelölésével az elem összes pontja kiválasztásra kerül, amennyiben például \textit{ALT} billentyű lenyomása közben történik.

Abban egészen biztos vagyok, hogy a szakdolgozat elkészítése során szerzett rengeteg hasznos ismeretet alkalmazni tudom majd a jövőben is.