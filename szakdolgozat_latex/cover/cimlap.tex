\pagestyle{empty} %a címlapon ne legyen semmi=empty, azaz nincs fejléc és lábléc

% A Miskolci Egyetem címere
{\large
\begin{center}
\vglue 1truecm
\textbf{\huge\textsc{Szakdolgozat}}\\
\vglue 1truecm
\includegraphics[width=4.8truecm, height=4truecm]{images/me_logo.png}\\
\textbf{\textsc{Miskolci Egyetem}}
\end{center}}

\vglue 1.5truecm %függõleges helykihagyás

% A szakdolgozat címe, akár több sorban is
{\LARGE
\begin{center}
\textbf{A szakdolgozat címe}
\end{center}}

\vspace*{2.5truecm}
% A hallgató neve, évfolyam, szak(ok), a konzulens(ek) neve
{\large
\begin{center}
\begin{tabular}{c}
\textbf{Készítette:}\\
Szakdolgozó Neve\\
Programtervező informatikus
\end{tabular}
\end{center}
\begin{center}
\begin{tabular}{c}
\textbf{Témavezető:}\\
Témavezető neve
\end{tabular}
\end{center}}
\vfill
% Keltezés: Hely, év
{\large
\begin{center}
\textbf{\textsc{Miskolc, 2020}}
\end{center}}

\newpage
