\Chapter{Bevezetés}

A \LaTeX\cite{latex}\cite{latexlamport}-et széles körben használják a tudományos életben tudományos dokumentumok közlésére és közzétételére számos területen, többek között a matematikában, az informatikában, a mérnöki tudományokban, a fizikában, a kémiában, a közgazdaságtanban, a nyelvészetben, a kvantitatív pszichológiában, a filozófiában és a politikatudományban. A \LaTeX\ a \TeX\ szövegszerkesztő programot használja a kimenet formázásához, és maga is a \TeX\ makrónyelvben íródott.

A többi szövegszerkesztő programmal ellentétben, amelyek a WYSIWYG (\textit{what-you-see-is-what-you-get}) elv szerint működnek, a szerző szövegfájlokkal dolgozik, amelyekben szövegesen, parancsokkal jelöli ki azokat a részeket vagy címsorokat, amelyeket egy szövegen belül másképp kell formázni. Mielőtt a \LaTeX\ rendszer a szöveget megfelelően beállítaná, fel kell dolgoznia a forráskódot. A \LaTeX\ által generált elrendezés nagyon tiszta, a képletkészlet pedig kifinomult. A \LaTeX\ különösen alkalmas az olyan terjedelmes munkákhoz, mint a szakdolgozatok és disszertációk, amelyeknek gyakran szigorú tipográfiai követelményeknek kell megfelelniük.

A tanulmányaim során megismerkedtem a JavaScript\cite{js} nyelvvel és egy könnyen elsajátítható programozási nyelvnek találtam. A JavaScript, gyakran JS rövidítéssel, az ECMAScript\cite{ecmascript} specifikációnak megfelelő programozási nyelv. A JavaScript magas szintű, gyakran futásidejű fordítású nyelv. Az ECMAScript 2015\cite{ecmascript6} (\textit{ES6}) bevezetésével egy jól használható  objektumorientált nyelv lett. A JavaScript kódot közvetlenül a weboldalakba lehet beágyazni, így a webböngésző gondoskodik a szkripteknek nevezett programok végrehajtásáról. Általában a JavaScript-et a HTML-űrlapokba beírt adatok vezérlésére, vagy a HTML-dokumentummal való interakcióra használják. Dinamikus alkalmazások, átmenetek, animációk létrehozására vagy adatok manipulálására is használható.

A szakdolgozatom egy online grafikus szerkesztő elkészítése és dokumentálása. A szerkesztő a \textit{TikZ} \LaTeX\ csomag nyelvi elemeire épül. A webes fejlesztés miatt a HTML5\cite{html5} és a JavaScript nyelv adja az alapokat. A rajzolási felületet a \textit{p5.js}\cite{p5js} nyújtja.

A következő oldalakban megismerkedhetünk a TikZ\cite{tikzmanual2} csomag telepítésével, a grafikus elemivel. A \LaTeX\ funkcióinak kiegészítésére a felhasználó harmadik féltől származó csomagokat tölthet be. Ezek, akárcsak a függvénykönyvtárak, további parancsokat biztosítanak, az egyszerű szimbólumoktól kezdve az összetett funkciókig. Ezt követően a már meglévő szerkesztők kerülnek jellemzésre, majd összehasonlításra különböző szempontok alapján. Ezek a feltételek kerülnek megfogalmazásra a készülő alkalmazással szemben, mint követelmények.

 A következő szegmens már a dokumentáció része az alkalmazásnak. Itt kerülnek kifejtésre a felhasznált függvénykönyvtárak, az elkészült alkalmazás felépítése, funkciói, használata, és a definiált osztályok leírása, a \textit{TikZ}, mint \LaTeX\ könyvtár és a \textit{p5.js}, mint JavaScript függvénykönyvtár eszközkészletében található eltérések. 
 
 A végezetül már az elkészült alkalmazással megvalósított ábrák kerülnek bemutatásra.
